\IfFileExists{stacks-project.cls}{%
\documentclass{stacks-project}
}{%
\documentclass{amsart}
}

% The following AMS packages are automatically loaded with
% the amsart documentclass:
%\usepackage{amsmath}
%\usepackage{amssymb}
%\usepackage{amsthm}

% For dealing with references we use the comment environment
\usepackage{verbatim}
\newenvironment{reference}{\comment}{\endcomment}
%\newenvironment{reference}{}{}
\newenvironment{slogan}{\comment}{\endcomment}
\newenvironment{history}{\comment}{\endcomment}

% For commutative diagrams you can use
% \usepackage{amscd}
\usepackage[all]{xy}

% We use 2cell for 2-commutative diagrams.
\xyoption{2cell}
\UseAllTwocells

% To put source file link in headers.
% Change "template.tex" to "this_filename.tex"
% \usepackage{fancyhdr}
% \pagestyle{fancy}
% \lhead{}
% \chead{}
% \rhead{Source file: \url{template.tex}}
% \lfoot{}
% \cfoot{\thepage}
% \rfoot{}
% \renewcommand{\headrulewidth}{0pt}
% \renewcommand{\footrulewidth}{0pt}
% \renewcommand{\headheight}{12pt}

\usepackage{multicol}

% For cross-file-references
\usepackage{xr-hyper}

% Package for hypertext links:
\usepackage{hyperref}

% For any local file, say "hello.tex" you want to link to please
% use \externaldocument[hello-]{hello}
\externaldocument[introduction-]{introduction}
\externaldocument[conventions-]{conventions}
\externaldocument[sets-]{sets}
\externaldocument[categories-]{categories}
\externaldocument[topology-]{topology}
\externaldocument[sheaves-]{sheaves}
\externaldocument[sites-]{sites}
\externaldocument[stacks-]{stacks}
\externaldocument[fields-]{fields}
\externaldocument[algebra-]{algebra}
\externaldocument[brauer-]{brauer}
\externaldocument[homology-]{homology}
\externaldocument[derived-]{derived}
\externaldocument[simplicial-]{simplicial}
\externaldocument[more-algebra-]{more-algebra}
\externaldocument[smoothing-]{smoothing}
\externaldocument[modules-]{modules}
\externaldocument[sites-modules-]{sites-modules}
\externaldocument[injectives-]{injectives}
\externaldocument[cohomology-]{cohomology}
\externaldocument[sites-cohomology-]{sites-cohomology}
\externaldocument[dga-]{dga}
\externaldocument[dpa-]{dpa}
\externaldocument[hypercovering-]{hypercovering}
\externaldocument[schemes-]{schemes}
\externaldocument[constructions-]{constructions}
\externaldocument[properties-]{properties}
\externaldocument[morphisms-]{morphisms}
\externaldocument[coherent-]{coherent}
\externaldocument[divisors-]{divisors}
\externaldocument[limits-]{limits}
\externaldocument[varieties-]{varieties}
\externaldocument[topologies-]{topologies}
\externaldocument[descent-]{descent}
\externaldocument[perfect-]{perfect}
\externaldocument[more-morphisms-]{more-morphisms}
\externaldocument[flat-]{flat}
\externaldocument[groupoids-]{groupoids}
\externaldocument[more-groupoids-]{more-groupoids}
\externaldocument[etale-]{etale}
\externaldocument[chow-]{chow}
\externaldocument[intersection-]{intersection}
\externaldocument[pic-]{pic}
\externaldocument[adequate-]{adequate}
\externaldocument[dualizing-]{dualizing}
\externaldocument[duality-]{duality}
\externaldocument[discriminant-]{discriminant}
\externaldocument[local-cohomology-]{local-cohomology}
\externaldocument[curves-]{curves}
\externaldocument[resolve-]{resolve}
\externaldocument[models-]{models}
\externaldocument[pione-]{pione}
\externaldocument[etale-cohomology-]{etale-cohomology}
\externaldocument[proetale-]{proetale}
\externaldocument[crystalline-]{crystalline}
\externaldocument[spaces-]{spaces}
\externaldocument[spaces-properties-]{spaces-properties}
\externaldocument[spaces-morphisms-]{spaces-morphisms}
\externaldocument[decent-spaces-]{decent-spaces}
\externaldocument[spaces-cohomology-]{spaces-cohomology}
\externaldocument[spaces-limits-]{spaces-limits}
\externaldocument[spaces-divisors-]{spaces-divisors}
\externaldocument[spaces-over-fields-]{spaces-over-fields}
\externaldocument[spaces-topologies-]{spaces-topologies}
\externaldocument[spaces-descent-]{spaces-descent}
\externaldocument[spaces-perfect-]{spaces-perfect}
\externaldocument[spaces-more-morphisms-]{spaces-more-morphisms}
\externaldocument[spaces-flat-]{spaces-flat}
\externaldocument[spaces-groupoids-]{spaces-groupoids}
\externaldocument[spaces-more-groupoids-]{spaces-more-groupoids}
\externaldocument[bootstrap-]{bootstrap}
\externaldocument[spaces-pushouts-]{spaces-pushouts}
\externaldocument[groupoids-quotients-]{groupoids-quotients}
\externaldocument[spaces-more-cohomology-]{spaces-more-cohomology}
\externaldocument[spaces-simplicial-]{spaces-simplicial}
\externaldocument[spaces-duality-]{spaces-duality}
\externaldocument[formal-spaces-]{formal-spaces}
\externaldocument[restricted-]{restricted}
\externaldocument[spaces-resolve-]{spaces-resolve}
\externaldocument[formal-defos-]{formal-defos}
\externaldocument[defos-]{defos}
\externaldocument[cotangent-]{cotangent}
\externaldocument[examples-defos-]{examples-defos}
\externaldocument[algebraic-]{algebraic}
\externaldocument[examples-stacks-]{examples-stacks}
\externaldocument[stacks-sheaves-]{stacks-sheaves}
\externaldocument[criteria-]{criteria}
\externaldocument[artin-]{artin}
\externaldocument[quot-]{quot}
\externaldocument[stacks-properties-]{stacks-properties}
\externaldocument[stacks-morphisms-]{stacks-morphisms}
\externaldocument[stacks-limits-]{stacks-limits}
\externaldocument[stacks-cohomology-]{stacks-cohomology}
\externaldocument[stacks-perfect-]{stacks-perfect}
\externaldocument[stacks-introduction-]{stacks-introduction}
\externaldocument[stacks-more-morphisms-]{stacks-more-morphisms}
\externaldocument[stacks-geometry-]{stacks-geometry}
\externaldocument[moduli-]{moduli}
\externaldocument[moduli-curves-]{moduli-curves}
\externaldocument[examples-]{examples}
\externaldocument[exercises-]{exercises}
\externaldocument[guide-]{guide}
\externaldocument[desirables-]{desirables}
\externaldocument[coding-]{coding}
\externaldocument[obsolete-]{obsolete}
\externaldocument[fdl-]{fdl}
\externaldocument[index-]{index}

% Theorem environments.
%
\theoremstyle{plain}
\newtheorem{theorem}[subsection]{Theorem}
\newtheorem{proposition}[subsection]{Proposition}
\newtheorem{lemma}[subsection]{Lemma}

\theoremstyle{definition}
\newtheorem{definition}[subsection]{Definition}
\newtheorem{example}[subsection]{Example}
\newtheorem{exercise}[subsection]{Exercise}
\newtheorem{situation}[subsection]{Situation}

\DeclareMathOperator{\Spec}{Spec}

\setcounter{section}{-1}
\theoremstyle{remark}
\newtheorem{remark}[subsection]{Remark}
\newtheorem{remarks}[subsection]{Remarks}

\numberwithin{equation}{subsection}

% Macros
%
\def\lim{\mathop{\rm lim}\nolimits}
\def\colim{\mathop{\rm colim}\nolimits}
\def\Spec{\mathop{\rm Spec}}
\def\Hom{\mathop{\rm Hom}\nolimits}
\def\Ext{\mathop{\rm Ext}\nolimits}
\def\SheafHom{\mathop{\mathcal{H}\!{\it om}}\nolimits}
\def\SheafExt{\mathop{\mathcal{E}\!{\it xt}}\nolimits}
\def\Sch{\textit{Sch}}
\def\Mor{\mathop{\rm Mor}\nolimits}
\def\Ob{\mathop{\rm Ob}\nolimits}
\def\Sh{\mathop{\textit{Sh}}\nolimits}
\def\NL{\mathop{N\!L}\nolimits}
\def\proetale{{pro\text{-}\acute{e}tale}}
\def\etale{{\acute{e}tale}}
\def\QCoh{\textit{QCoh}}
\def\Ker{\mathop{\rm Ker}}
\def\Im{\mathop{\rm Im}}
\def\Coker{\mathop{\rm Coker}}
\def\Coim{\mathop{\rm Coim}}

%
% Macros for moduli stacks/spaces
%
\def\QCMfunctor{\mathcal{QC}\!{\it oh}}
\def\CMfunctor{\mathcal{C}\!{\it oh}}
\def\Spacesstack{\mathcal{S}\!{\it paces}}
\def\Quotfunctor{{\rm Quot}}
\def\Hilbfunctor{{\rm Hilb}}
\def\Curvesstack{\mathcal{C}\!{\it urves}}
\def\Polarizedstack{\mathcal{P}\!{\it olarized}}
\def\Complexesstack{\mathcal{C}\!{\it omplexes}}
% \Pic is the operator that assigns to X its picard group, usage \Pic(X)
% \Picardstack_{X/B} denotes the Picard stack of X over B
% \Picardfunctor_{X/B} denotes the Picard functor of X over B
\def\Pic{\mathop{\rm Pic}\nolimits}
\def\Picardstack{\mathcal{P}\!{\it ic}}
\def\Picardfunctor{{\rm Pic}}
\def\Deformationcategory{\mathcal{D}\!{\it ef}}
\def\CMfunctor{\text{CM}}



% OK, start here.
%
\begin{document}

\title{The CM functor}


\maketitle


\section{The (Honsen?) functor}
\label{section-cm}

\noindent
In this section we prove the (Honsen?) functor is an algebraic space.\\

\begin{situation}
\label{situation-cm}
Let $S$ be a scheme. Let $f : X \to B$ be a morphism of algebraic spaces
over $S$. Assume that $f$ is separated and of finite presentation.
For any scheme $T$ over $B$ we will denote $X_T$ the base change
of $X$ to $T$. Given such a $T$ we set:
\begin{align*}
\CMfunctor_{X/B}(T) = & (h, C \to T, g)  \text{ where } 
h \text{ is a morphism } h : T \to B \text{ and } g \text{ is a morphism } g : C \to X \\ & \text{satisfying the following properties}
\end{align*}
\begin{enumerate}
\item\label{situation-cm-diagram} The diagram below is commutative: 
$$
\xymatrix{
C \ar[r]_g \ar[d] & X \ar[d] \\
T \ar[r]^h & B
}
$$
\item\label{situation-cm-morphism} $C$ is an algebraic space and $C \to T$ is a Cohen-Macauley morphism of relative dimension 1 which is flat, proper, and of finite presentation.
\item\label{situation-cm-finite} The induced morphism $(g') : C \to X_T$ is finite
\item\label{situation-cm-open} There is an open $U \subset X_T$ where
\begin{enumerate}
\item $(g')^{-1}(U) \to U$ is a closed immersion
\item For every $t \in T$, the fiber $\left((g')^{-1}(U)\right)_t$ is dense in $C_t$
\end{enumerate}
\end{enumerate}
\end{situation}

\begin{lemma}\label{cm-fibered-in-setoids}
Let $S$ be a scheme.
Let $T$ a scheme over $S$.
Let $f \colon X \to B$ be a separated and finitely presented morphism of algebraic spaces over $S$.
Then for any two elements $\gamma_1 = (h_1,C_1 \to T,g_1)$ and $\gamma_2 = (h_2,C_2 \to T,g_2)$ of $\CMfunctor_{X/B}(T)$.
Then there is at most one isomorphism $\gamma_1 \to \gamma_2$.
\end{lemma}
\begin{proof}
Let $a,b \colon \gamma_1 \to \gamma_2$ be two isomorphisms.
Then $h_1 = h_2$ and $a,b$ correspond to isomorphisms $\bar{a},\bar{b} \colon C_1 \xrightarrow{\sim} C_2$ making the following diagram commutative:
\[ \xymatrix{
  C_1 \ar[r]^{g_1} \ar[d]_{\bar{a},\bar{b}} & X \ar@{=}[d] \\
C_2 \ar[r]^{g_2} & X
} \]
Let $U \subset X_T$ be an open subset satisfying condition~(\ref{situation-cm-open}) from Situation~\ref{situation-cm} for $\gamma_2$.
Let $U' = (g'_2)^{-1}(U)$ and $V' = \bar{a}^{-1}(U') \cap \bar{b}^{-1}(U') \subset C_1$.
Then $g'_2|_{U'}$ is a closed immersion of algebraic spaces, hence a monomorphism (Lemma~Tag 042R).
As $g'_2|_{U'} \circ \bar{a}|_{V'} = g'_2|_{U'} \circ \bar{b}|_{V'}$, we have $\bar{a}|_{V'} = \bar{b}|_{V'}$.

On the other hand, $U' \subset C_2$ is scheme theoretically dense.
Since $\bar{a}$ and $\bar{b}$ are isomorphisms, $\bar{a}^{-1}(U') \cap \bar{b}^{-1}(U') \subset C_1$ is scheme theoretically dense by Lemma~Tag 0837.
Lemma~Tag 084N shows that $\bar{a} = \bar{b}$, and thus $a = b$.
\end{proof}

\begin{lemma}
	In Situation \ref{situation-cm}, the diagonal morphism 
	\[\Delta:\CMfunctor_{X/B}\rightarrow\CMfunctor_{X/B}\times\CMfunctor_{X/B}\]
	is representable by algebraic spaces and quasicompact.
	\label{cm-rep-diagonal}
\end{lemma}
\begin{proof}
	Let $( (h_0,C_0\rightarrow T, g_0) ,( h_1, C_1\rightarrow T, g_1))\in \CMfunctor_{X/B}\times\CMfunctor_{X/B}(T)$.
	First, observe that the pair induces an element $(C_0\rightarrow T,C_1\rightarrow T)$ in $\textit{B-curves}\times \textit{B-curves}(T)$.
	Because $\textit{B-curves}$ is an algebraic stack, its diagonal pulled back to $T$ is an algebraic space $U$.

	At this point, we have
	\[\xymatrix{
		C\ar[d] & \\
		U\ar[r] & T
	}\]
	and two $T$-morphisms $C\xrightarrow{g_0,g_1} X$.
	Then using the fact that $Mor(C,X)$ is an algebraic stack, the equalizer defines an algebraic space $V$ over $T$ with the property that
	\[\xymatrix{
		C\ar[rr]\ar[dr]\ar[dd]&&C_0\ar[dd]|!{[dl];[dr]}\hole\ar[dr]^{g_0}&\\
		&C_1\ar[rr]^(.25){g_1}\ar[dr]& & X\ar[d]\\
		V\ar[rr]& & T\ar[r]^{h} & B
	}
	\]
	is commutative.

	Because $C\cong C_0|_V$, properties (2) and (3) hold because they are stable under base change.
	For property (4), simply pull back the open set in $C_0$ to $C$.
	This demonstrates that $\Delta$ is representable.

	Now, we note that $U\rightarrow T$ is quasicompact as it is the pullback of the diagonal of $\text{B-curves}$ to $T$ (Lemma Tag 0DSQ).
	Next, $V\rightarrow U$ is quasicompact as it is the pullback of the diagonal of $Mor(C,X)$ to $U$ (Lemma Tag 0DPM).
	Thus, $V\rightarrow T$ is quasicompact.
\end{proof}

\begin{lemma}
	Let $S$ be a scheme.
Let $f \colon X \to B$ be a separated and finitely presented morphism of algebraic spaces over $S$.
Then the functor $\CMfunctor_{X/B}$ is a sheaf for the fppf topology.
	\label{cm-is-sheaf}
\end{lemma}

\begin{proof}
Let $T$ be a scheme over $S$.
Let $\{ T_i \to T \}_{i \in I}$ an fppf covering of $T$.
We must prove that
\[ \xymatrix{
  \CMfunctor_{X/B}(T) \ar[r] & \prod\nolimits_{i\in I} \CMfunctor_{X/B}(T_i) \ar@<1ex>[r]^-{\text{pr}_0^*} \ar@<-1ex>[r]_-{\text{pr}_1^*} & \prod\nolimits_{(i_0, i_1) \in I \times I} \CMfunctor_{X/B}(T_{i_0} \times_T T_{i_1})
} \]
is an equalizer diagram.
Injectivity is immediate from descent for algebraic spaces, see Lemma~Tag 0ADT, and because $X$ and $B$ are, per definition, sheaves for the fppf topology.

Let us proceed to exactness in the middle.
Suppose we are given $\gamma_i = (h_i,C_i \to T,g_i) \in \CMfunctor_{X/B}(T_i)$ such that 
\[ \gamma_i|_{T_i \times_T T_j} = \gamma_j|_{T_i \times_T T_j} \]
for all $i,j \in I$.
In particular, there exist isomorphisms $\varphi_{ij} \colon C_i \times_T T_j \xrightarrow{\sim} T_i \times_T C_j$ such that $g_j|_{T_i \times_T T_j} \circ \varphi_{ij} = g_i|_{T_i \times_T T_j}$.
By Lemma~\ref{cm-fibered-in-setoids}, $(C_i,\varphi_{ij})$ forms a descent datum for algebraic spaces relative to $\{T_i \to T\}$.
As fppf descent data for algebraic spaces of finite type are effective, Lemma~Tag 04TR and Lemma~Tag 04U0, there is an algebraic space $C \to T$ such that $C \times_T T_i = C_i$; see also the proof of condition (3) in Lemma~Tag 0D1G.
Further, since algebraic spaces are sheaves in the fppf topology, the morphisms $g_i \colon C_i \to X$ and $h_i \colon T_i \to B$ glue to morphisms $g \colon C \to X$ and $h \colon T \to B$, respectively.

Set $\gamma = (h,C\to T,g)$.
It remains to show that $\gamma \in \CMfunctor_{X/B}(T)$.
Property~(\ref{situation-cm-diagram}) holds because commutativity of the diagram can be checked fppf locally on $C$.

The morphism $C \to T$ is of relative dimension $1$, flat, proper, and of finite presentation by Lemma~Tag 02VJ, Lemma~Tag 02L2, Lemma~Tag 02L1, and Lemma~Tag 02L0.
Moreover, for a flat morphism, being Cohen-Macaulay can be checked on fibers, see Lemma~Tag 0E0W. That establishes property~(\ref{situation-cm-morphism}).

For property~(\ref{situation-cm-finite}), the finiteness of $(g') \colon C \to X_T$ can again be seen fppf locally on $X_T$ from the finiteness of the $(g')|_{T_i} = (g'_i) \colon C_i \to X_{T_i}$ by Lemma~Tag 02LA.

Finally, let $U = X_T \setminus (\text{Supp}\Coker(\mathcal{O}_{X_T} \to (g')_*\mathcal{O}_C))$.
This is the largest open subset for which $(g')^{-1}(U) \to U$ is an closed immersion.
For $t \in T$, choose $i \in I$ such that $t \in \Im(T_i \to T)$.
Let $\tilde{t} \in T_i$ be a point mapping to $t$.
Then $((g')^{-1}(U))_t = ((g'_i)^{-1}(U))_{\tilde{t}}$ is dense in $C_t = (C_{i})_{\tilde{t}}$, yielding property~(\ref{situation-cm-open}).
\end{proof}


\begin{remark}
	\label{remark-cm-base-change}
	In Situation \ref{situation-cm} the rule
$(T, g, \mathcal{F}) \mapsto (T, g)$ defines a morphism
$$
\CMfunctor_{X/B} \longrightarrow B
$$
of sheaves
( Algebraic Stacks, Section \ref{algebraic-section-split}, and
Examples of Stacks, Section
\ref{examples-stacks-section-stack-associated-to-sheaf}).
Let $B' \to B$ be a morphism of
algebraic spaces over $S$.
Set $X' = X \times_B B'$.
We obtain a sheaf $\CMfunctor_{X'/B'}$ on  $(\Sch/S)_{fppf}$
associated to the base change $f' : X' \to B'$. In this situation
the diagram
$$
\vcenter{
\xymatrix{
\CMfunctor_{X'/B'} \ar[r] \ar[d] & \CMfunctor_{X/B} \ar[d] \\
B' \ar[r] & B
}
}
$$
is fibre product square. This trivial remark
will occasionally be useful to change the base algebraic space.
\end{remark}
\begin{lemma}
    In Situation~\ref{situation-cm}, $\CMfunctor_{X/B}$ is limit-preserving.
	\label{lemma-cm-limits}
\end{lemma}
\begin{proof}
Let $T = \lim_{i \in I} T_i$ be a directed limit of affine $B$-schemes. We need to show that
$$
\CMfunctor_{X/B}(T) = \colim_{i \in I} \CMfunctor_{X/B}(T_i)
$$
Using Tag~0D1I (limit preservation for the stack of spaces), this reduces to the following assertion. For any $0 \in I$ and a commutative diagram
$$
\xymatrix{
C_0 \ar[r]^{g_0} \ar[d] &X \ar[d] \\
T_0 \ar[r]^{h_0} & B
}
$$
    denote the base change to $T_i$ with $i \geq 0$ by $(h_i,C_i \rightarrow T_i,g_i)$, and the base change to $T$ by $(h,C \rightarrow T,g)$. We need to show that if $(h,C \rightarrow T,g) \in \CMfunctor_{X/B}(T)$, then there exists an $i \geq 0$ with $(h_i, C_i \rightarrow T_i, g_i) \in \CMfunctor_{X/B}(T_i)$. From Tag~084Z (finiteness behaves nicely w.r.t.\ limits) and Tag~01EF (stack of relative dimension 1 CM curves is algebraic), we find an $i \geq 0$ so that~\eqref{situation-cm-morphism} and~\eqref{situation-cm-finite} hold.

\medskip\noindent
We still need to show the existence of an $i \geq 0$ for which the set
$$
U_i = X_{T_i} \setminus \text{Supp}(\Coker(\mathcal{O}_{X_{T_i}} \rightarrow g'_{i,*} \mathcal{O}_{C_i}))
$$
has fiberwise dense preimage in $C_i$. Since $C_i \rightarrow T_i$ has relative dimension $1$, this is equivalent to requiring that $C_i \setminus (g_i')^{-1} (U_i)$ has relative dimension $0$ over $T_i$. After showing that the $U_i$ are stable under taking preimages, this follows from Tag~0D4K (relative dimension behaves nicely w.r.t.\ limits). \\
The stability of the $U_i$ under taking preimages boils down to the following. Given cartesian squares
$$
\xymatrix{
C_0 \ar[r]^{g_0} \ar[d]^{g_0'} &C_1 \ar[d]^{g_1'} \\
X_{T_0} \ar[d] \ar[r]^{b} & X_{T_1} \ar[d] \\
T_0 \ar[r] &T_1
}
$$
with $g'_1$ finite, the preimage of $(g'_1)^{-1} \text{Supp}( \Coker(\mathcal{O}_{X_{T_1}} \rightarrow g'_{1,*} \mathcal{O}_{C_1}))$ under $b$ is equal to $(g'_0)^{-1} \text{Supp}( \Coker(\mathcal{O}_{X_{T_0}} \rightarrow g'_{0,*} \mathcal{O}_{C_0}))$. This follows from Tag~07U8 (base change for affine morphisms) and Tag~056J (preimage of support of finite-type quasi-coherent sheaf), which applies because the finiteness of $g_1'$ implies that $g_{1,*}' \mathcal{O}_{C_1}$ is finite-type quasi-coherent {\bfseries Reference?}.
\end{proof}


\begin{lemma}
In Situation \ref{situation-cm}, let 
$$
\xymatrix{
Z \ar[r] \ar[d] & Z' \ar[d] \\
Y \ar[r] & Y'}
$$

be a pushout in the category of schemes over $B$ where $Z\to Z'$ is a nilpotent thickening and $Z \to Y$ is affine. Then the natural map
$$
\CMfunctor_{X/B}(Y') \longrightarrow \CMfunctor_{X/B}(Y) \times_{\CMfunctor_{X/B}(Z')} \CMfunctor_{X/B}(Z)
$$ is a bijection.

\label{lemma-cm-RS-star}
\end{lemma}

\begin{proof}

First, consider the natural map 
$$
\textit{CM}_{X/B} \longrightarrow \textit{B-Curves},
$$ on $(\Sch/S)_{fppf}$ which maps the triple $(g, C \to T, h) \mapsto (C \to T, h)$ (see Remark Tag 0D54). \\

For a scheme $T$, the image of the map $\CMfunctor(T) \to \text{B-Curves}(T)$ is the subcategory of those curves which are Cohen-Macauley over $T$. Let $F$ be the subfunctor of $\textit{B-Curves}$ which are also Cohen-Macauley. This is well defined as the base change of a Cohen-Macauley morphism is Cohen-Macauley (More on Morphisms of Spaces tag 0E0Z) and for each scheme $T$, $F(T)$ will be a full subcategory of $\CMfunctor(T)$ by (Tag 0E0X). For any square as above, $$
F(Y') \longrightarrow F(Y) \times_{F(Z')} F(Z)
$$
is an equivalence of categories. To show this, note that it is sufficient to check that for any morphism $C \to Y'$, we have $C \to Y'$ is Cohen-Macauley if and only if $C_Y \to Y$ and $C_Z' \to Z'$ are Cohen-Macauley, as it is a subfunctor of $\textit{B-Curves}$, which satisfies $(RS^*)$ by Tag 0D56. One direction holds because the base change of a Cohen-Macauley morphism is Cohen-Macauley (More on Morphisms of Spaces tag 0E0Z). The other direction holds because this condition is checked on the fibers, and the fibers of $C_Y \to Y$ and $C \to Y'$ are the same because $y \to Y'$ is a thickening by Tag 07RT.\\

The preimage of any such curve $C\to T$ is equivalent to the subset of $\Mor_T(C,X_T)(T)$ consisting of those morphisms which are finite and satisfy condition $(4)$. We define a subfunctor $G_C$ of $\Mor_T(C,X_T)$ consisting of such morphisms. As $C \to T$ and $X_T \to T$ satisfy the hypotheses of (Tag 0D1C), $\Mor_T(C,X_T)$ is an algebraic space, and hence satisfies $(RS^*)$. The subfunctor of finite morphisms satisfies $(RS^*)$, following from tag 03ZO and 09ZY. Thus to show that given any square as above $$
G_C(Y') \longrightarrow G_C(Y) \times_{G_C(Z')} G_C(Z)
$$
is a bijection we have reduced to the following statement: given a finite morphism $C \to X_{Y'}$, then there exists a $U_1\subset X_{Y'}$ satisfying $(4)$ if and only if there exists a $U_2 \subset X_Y$ and $U_3 \subset X_{Z'}$ satisfying $(4)$ for their respective morphisms. Given $U_1$, construct $U_2$ and $U_3$ as in the definition of the functor from the morphisms $Y \to Y'$ and $Z' \to Y'$. Given $U_2$, construct $U_1$ as in (tag 05ZP). For $(4a)$, one implication follows from the fact that open and closed immersions are preserved under base change (tag 02YW), the other by (tag 09ZZ) and (tag 0CGA). For $(4b)$, one direction follows by discussion on the functoriality of $\CMfunctor$. The other follows from the fact that thickenings are universal homeomorphisms, from e.g. (tag 054M).\\

We now proceed with the proof. As our functor is setoid valued, we argue as in the category of sets.\\

Consider the following diagram:
$$
\xymatrix{
\CMfunctor_{X/B}(Y') \ar[r] \ar[d]_\alpha & \CMfunctor_{X/B}(Y)\times_{\CMfunctor_{X/B}(Z)} \CMfunctor_{X/B}(Z') \ar[d] _\beta\\
F(Y') \ar[r] & F(Y)\times_{F(Z)} F(Z')}
$$

We wish to show the arrow on the top is a bijection.\\

Let $x,y \in \CMfunctor_{X/B}(Y')$ which map to the same object. Then $\alpha(x) = C_1 \to Y'$ and $\alpha(y) = C_2 \to Y'$ are isomorphic curves, say by isomorphism $\phi: C_1 \to C_2$ commuting with the structure maps. Lying above $\alpha(x)$ is the subcategory $G_{C_1}(Y')$. As $x$ and $y$ map to the same object and $G_{C_i}$ satisfy $(RS^*)$ for $i=1,2$, there are unique maps $C_1 \to X_{Y'}$ and $C_2 \to X_{Y'}$. As in Lemma \ref{cm-fibered-in-setoids}, this implies that $C_1=C_2$ and so $x=y$.

Let $z \in \CMfunctor_{X/B}(Y)\times_{\CMfunctor_{X/B}(Z)} \CMfunctor_{X/B}(Z')$. Represent this in the fiber product by $(z_1,z_2)$, where . Then, as the arrow on the bottom is a bijection, there is a curve $C \to Y'$ mapping to $\beta(z)$. Then there is a unique morphism $C \to X_{Y'} \in G_C(Y')$ restricting to the image of $z_1$ and $z_2$ in $G_{C_1}(Y)$ and $G_{C_2}(Z')$ respectively. Putting these together we find a curve in $\CMfunctor_{X/B}(Y')$ mapping to $z$ and we are done.
\end{proof}

\begin{lemma}
	For every field $k$ and every $x_0\in \CMfunctor_{X/B}(k)$, $T_{k,x_0}\CMfunctor_{X/B}$ is finite-dimensional.
	\label{lemma-cm-tangent-space}
\end{lemma}
\begin{proof}
	First, we may assume that $S=B$.

	An element in $ T_{k,x_0}\CMfunctor_{X/S}$ is simply an element
	\[(h',C'\rightarrow \Spec(k[\epsilon]),g')\in \CMfunctor_{X/S}(\Spec(k[\epsilon]))\]
	which restricts to $x_0$ under 
	\[\CMfunctor_{X/S}(\Spec(k[\epsilon]))\rightarrow\CMfunctor_{X/S}(\Spec(k)).\]
	Note that $h'$ is $h$ composed with the natural morphism $\Spec(k[\epsilon])\rightarrow\Spec(k)$ because we are working with $S$-schemes.

	Observe that there is an obvious forgetful morphism
	\[\CMfunctor_{X/S}\rightarrow Spaces'_{fp,flat,proper}\]
	which sends $(h,C\rightarrow T,g)$ to $C$.

	This induces a morphism of tangent spaces which sends $(C',g')$ to $C'$.

	Clearly, a tangent vector is in the kernel of this morphism if and only if $C'\cong C\times_{\Spec(k)} \Spec(k[\epsilon])$.
	
	Recall that $Mor(C,X_k)$ associates to a test scheme $T\rightarrow S$ the set of morphisms $C_T\rightarrow X_k|_T$.
	The tangent space at $g$ is the set of commutative diagrams
	\[\xymatrix{
		&C\ar[rr]^{g}\ar[dr]|!{[d];[rr]}\hole\ar[dl] & & X_k\ar[dll]\ar[dl]\\
		C\times_{\Spec(k)}\Spec(k[\epsilon])\ar[r]\ar[dr]&X_k\times_{\Spec(k)}\Spec(k[\epsilon])\ar[d]& \Spec(k)\ar[dl]\ar[d] & \\
		&\Spec(k[\epsilon])\ar[r] & S&
	}\]
	Clearly, a morphism $C\times_{\Spec(k)}\Spec(k[\epsilon])\rightarrow X_k$ extending $g$ can be composed with the natural map $X_k\rightarrow X_k\times_{\Spec(k)}\Spec(k[\epsilon])$ to uniquely produce such a diagram.

	This implies that $g':C\times_{\Spec(k)}\Spec(k[\epsilon])\rightarrow X_k$ induces a tangent vector in $T_gMor(C,X_k)$.

	Thus, the kernel of
	\[T_{k,x_0}\CMfunctor_{X/S}\rightarrow T_{C}Spaces'_{fp,flat,proper}\]
	is included in the finite-dimensional tangent space $T_gMor(C,X_k)$.

	Altogether, we conclude that $T_{k,x_0}\CMfunctor_{X/S}$ is finite-dimensional, as claimed.
	\end{proof}
	\begin{remark}
		Strictly speaking, in the proof of Lemma \ref{lemma-cm-tangent-space}, we describe a vector space which merely contains $T_{k,x_0}\CMfunctor_{X/S}$--we have ignored data like the generic embedding condition.
		This is adequate however, because our goal is merely to demonstrate that the actual tangent space is finite-dimensional.
	\end{remark}


\begin{lemma}
	\label{lemma-cm-existence}
	Fix a complete Noetherian local $\mathcal{O}_S$-algebra $A$ and a collection of $\Spec A/m_i$ points of $CM_{X/B}$ for all $i$ which are compatible, call it $(h_i,C_i,g_i)$. Then there exists a $\Spec A$ point of $CM_{X/B}$ which restricts to $(h_i,C_i,g_i)$. 
\end{lemma}

\begin{proof}

We view the data of $(h,C, g)$ as a coherent algebra on $X_T$ by considering $(g_T)_*\mathcal{O}_{X_T}$. Conversely, given a coherent algebra $\mathcal{A}$ on $X_T$ that is $T$-flat with proper support over $T$, we take the relative spectrum of $\mathcal{A}$. Let $(h_i,C_i,g_i)$ be a compatible system of $\Spec A/m^n$ points of $CM_{X/B}$ and consider the corresponding coherent algebras $\mathcal{A}_i$ on $X_i=X \times_B  \Spec A/m^i$. 
\\

The underlying coherent sheaf and its algebra structure can be described by morphisms $\mathcal{A}_i \otimes \mathcal{A}_i \to \mathcal{A}_i$ and the unit $\mathcal{O}_{X_{A/m^i}} \to \mathcal{A}_i$. That this yields an algebra structure is expressed by the commutativity of certain diagrams. By Lemma 08BE these diagrams correspond to diagrams in $Coh_{X_A/\Spec A}$. As such we obtain a coherent algebra $\mathcal{A}$ on $X_A=X \times_B \Spec A$ whose relative spectrum is the candidate for algebraizing the formal point $(h_i,C_i,g_i)$ corresponding to $(\mathcal{A}_i)$. 
\\

It remains to show that $C=\underline{\Spec}_X(\mathcal{A})$ is a $\Spec A$-point of $CM_{X/B}$. Since it arises as the relative spectrum of a coherent algebra with proper support over $\Spec A$, $C \to X_{\Spec A}$ is finite with proper support of finite presentation over $\Spec A$. Since $C_i$ is $\Spec A/m^i$-flat for each $i$ Lemma 0523 shows $C$ is $\Spec A$-flat. To verify that $g_A: C \to X_A$ is a Cohen-Maucalay morphism use the fact that the closed fiber of $C \times_B \Spec A/m \to \Spec A/m$ is a Cohen-Maucalay morphism and Lemma 045U to conclude. To see that $C$ is a relative curve over $\Spec A$ apply Lemma 02NM, the fact that $C$ is proper over $\Spec A$, and that the closed fiber is relative dimension $1$. 
\\

It remains to show the locus where the morphism $C_T \to X_A$ is a closed immersion is dense on the fibers. Note that this locus is open as it can be identified with the complement of $\text{Supp}(\text{Coker}[O_{X_A} \to \mathcal{A}])$ and the latter is closed because it is the support of a coherent sheaf. Since pullback is right exact and taking supports is compatible with restriction to a closed subscheme (Lemma 00L3) we obtain 

\[\text{Supp}(\text{Coker}[O_{X_A} \to \mathcal{A}])|_{\Spec A/m}=\text{Supp}(\text{Coker}[O_{X_{A/m}} \to \mathcal{A}_0]))\]

and the latter is dense when pulled back to $C_0=C \times \Spec A/m$. This implies $g_A^{-1}(U)$ is dense in the closed fiber $C_0 \subset C$ and now we must conclude that it is dense in every fiber of $C$. Let $p \subset A$ be a prime ideal, by modding out and pulling back $g_A: C \to X_A$ we may suppose that $p=(0)$ is the generic point of $A$. 
\\

If there is a generic point of the generic fiber $\eta \in C \times_B \Spec A_{(0)}$ not in $g_A^{-1}(U)$ then the closure $\overline{\{\eta\}}$ is disjoint from $g_A^{-1}(U)$. Consider the composition $Y=\overline{\{\eta\}} \subset X_A \to \Spec A$ where $Y$ has the reduced induced structure. Then by Lemma 02FZ the set $U_0=\{y \in Y|\dim_y Y_{f(y)} = 0 \}$ is open in $Y$. Since $g_A^{-1}(U)$ contains all generic points of the closed fiber the closed fiber of $Y$ must have dimension $<1$. It follows that $U_0$ contains all of $Y \times_B \Spec {A/m}$ and therefore $Y \subset U_0$. Indeed, if $y \in Y$ then its closure must have a point in $Y \times \Spec A/m^i$ because $Y \to \Spec A$ is proper and $A$ is local. This contradicts the fact that $C \to \Spec A$ is a relative curve. It follows that $g_A^{-1}(U)$ contains all the generic points of the generic fiber, as desired. It follows that the resulting $g_A: C \to X_A$ is a $A$-point of $CM_{X/B}$ which algebraizes $(h_i,C_i,g_i)$. 

\end{proof}


Let $(A_n)$ be a inverse system of rings with surjective transition maps which have locally nilpotent kernels. Take $A=\lim A_n$, to show the strong form of formal effectiveness we prove that if $CM_{X/B}$ admits a compatible system of $\Spec A_n$-points, $\zeta_n \in CM_{X/B}(\Spec A_n)$, then it admits a $\Spec A$-point $\zeta \in CM_{X/B}(\Spec A)$ which extends the $(\zeta_n)$. 

\begin{lemma}[Strong Effectiveness] Let $A=\lim A_n$ and $(\zeta_n)$ be as above. Then there exists a $\zeta \in CM_{X/B}(\Spec A)$ which extends the $(\zeta_n)$. 
	\label{lemma-cm-strong-effectiveness}
\end{lemma}
\begin{proof}
We view the data of $(h,C, g) \in CM_{X/B}(T)$ as a quasicoherent algebra of finite presentation (as a $\mathcal{O}_T$-module) on $X_T$ by considering $(g_T)_*\mathcal{O}_{X_T}$. Conversely, given a quasicoherent algebra of finite presentation $\mathcal{A}$ on $X_T$ that is $T$-flat with proper support over $T$, we take the relative spectrum of $\mathcal{A}$. Let $(h_i,C_i,g_i)$ be a compatible system of $\Spec A_n$ points of $CM_{X/B}$ and consider the corresponding quasicoherent algebras $\mathcal{A}_i$ on $X_i=X \times_B  \Spec A_i$.
\\

First we algebraize the underlying quasicoherent module structure of the $(\mathcal{A}_n)$ to $\Spec A$ using Lemma 0CXB, call the resulting quasicoherent sheaf $\mathcal{A}$. It is flat over $\Spec A$, of finite presentation and has proper support over $\Spec A$. Next we will lift the algebra structure i.e. we need a $\Spec A$ point, $(m,u)$ of $Hom(\mathcal{A} \otimes \mathcal{A}, \mathcal{A}) \times Hom(\mathcal{O}_{X_A},\mathcal{A})$ which yields a multiplication and a section which are commutative, associative and so that the section behaves as a unit element with respect to this multiplication. 

That $(m,u)$ yields an algebra structure is expressed by the commutativity of certain diagrams whose arrows naturally depend only on the choice of multiplication and unit e.g. associativity is described as two different morphisms 
\[m \circ (m \otimes 1), m \circ (1 \otimes m): \mathcal{A} \otimes \mathcal{A} \otimes \mathcal{A} \to \mathcal{A}\]
being equal. Given a map $\Hom(\mathcal{A} \otimes \mathcal{A}, \mathcal{A})$ these are two ways to get a map in $Hom(\mathcal{A} \otimes \mathcal{A} \otimes \mathcal{A}, \mathcal{A})$. i.e. we have a morphism of sheaves
\[Hom(\mathcal{A} \otimes \mathcal{A}, \mathcal{A}) \to Hom(\mathcal{A} \otimes \mathcal{A} \otimes \mathcal{A}, \mathcal{A}) \times Hom(\mathcal{A} \otimes \mathcal{A} \otimes \mathcal{A}, \mathcal{A})\]
and the pullback of the diagonal is a closed locus which consists of those maps $\mathcal{A} \otimes \mathcal{A} \to \mathcal{A}$ satisfying the associativity law. 
\\

By the exact same argument, the other diagrams which express the fact that a map $\mathcal{A} \otimes \mathcal{A} \to \mathcal{A}$ is an algebra-map also introduce a closed condition (scheme-theoretically) in $Hom(\mathcal{A} \otimes \mathcal{A}, \mathcal{A}) \times Hom(\mathcal{O},\mathcal{A})$. Details omitted. 
\\

The scheme-theoretic intersection of these closed subschemes is exactly the locus of $Hom(\mathcal{A} \otimes \mathcal{A}, \mathcal{A}) \times Hom(\mathcal{O},\mathcal{A})$ consisting of multiplication maps and units turning $\mathcal{A}$ into a algebra, call it $Y_{alg}$. 
\\

Since $\mathcal{A}|_{X_{\Spec A_n}}$ has algebra structures for every $n$ these give us $\Spec A_n$ points of $Y_{alg} \subset Hom(\mathcal{A} \otimes \mathcal{A}, \mathcal{A}) \times Hom(\mathcal{O},\mathcal{A})$. By Lemma 08JT we know $Hom(\mathcal{A} \otimes \mathcal{A}, \mathcal{A}) \times Hom(\mathcal{O},\mathcal{A})$ and therefore $Y_{alg}$ is affine. So because we have compatible maps $H^0(Y_{alg},\mathcal{O}_{Y_{alg}}) \to A_n$ for every $n$ they factor as $H^0(Y_{alg},\mathcal{O}_{Y_{alg}}) \to A = \lim A_n \to  A_n$ and so we obtain $\Spec A_n \to \Spec A \to Y_{alg}$. Unwinding this, this says there is a unique lift of the algebra structure on $\mathcal{A}$ from that of the $\mathcal{A}_n$. 
\\

It remains to show that $C=\underline{\Spec}_X(\mathcal{A})$ is a $\Spec A$-point of $CM_{X/B}$. Since it arises as the relative spectrum of a finitely presented quasicoherent algebra which is $\Spec A$-flat with proper support over $\Spec A$, $C \to X_{\Spec A}$ is finite with proper support of finite presentation over $\Spec A$. Let $A \to A_0$ be natural map, it is surjective with kernel $I \subset A$. By Lemma 0CT7 we know that $I$ is a radical ideal, i.e. it is contained in all the maximal ideals of $A$. To verify that $g_A: C \to X_A$ is a Cohen-Maucalay morphism use the fact that the fiber product of $C \times_B \Spec A_0 \to \Spec A_0$ is a Cohen-Maucalay morphism and Lemma 045U to conclude. To see that $C$ is a relative curve over $\Spec A$ apply Lemma 02NM, the fact that $C$ is proper over $\Spec A$, and that the fiber product $C \times_A \Spec A/I$ is relative dimension $1$ over $\Spec A/I$. 
\\

To finish we shall show the locus where the morphism $C \to X_A$ is a closed immersion is dense on the fibers. Note that this locus is open as it can be identified with the complement of $\text{Supp}(\text{Coker}[O_{X_A} \to \mathcal{A}])$ and the latter is closed because it is the support of a finitely presented quasicoherent sheaf. Since pullback is right exact and taking supports is compatible with restriction to a closed subscheme (Lemma 00L3) we obtain 

\[\text{Supp}(\text{Coker}[O_{X_A} \to \mathcal{A}])|_{\Spec A/I}=\text{Supp}(\text{Coker}[O_{X_{A/I}} \to \mathcal{A}_0]))\]

and the latter is dense when pulled back to $C_0=C \times \Spec A/I$. This implies $g_A^{-1}(U)$ is dense in the closed fiber $C_0 \subset C$ and now we must conclude that it is dense in every fiber of $C$. Let $p \subset A$ be a prime ideal, by modding out and pulling back $g_A: C \to X_A$ we may suppose that $p=(0)$ is the generic point of $A$. 
\\

If there is a generic point of the generic fiber $\eta \in C \times_B \Spec A_{(0)}$ not in $g_A^{-1}(U)$ then the closure $\overline{\{\eta\}}$ is disjoint from $g_A^{-1}(U)$. Consider the composition $Y=\overline{\{\eta\}} \subset X_A \to \Spec A$ where $Y$ has the reduced induced structure. Then by Lemma 02FZ the set $U_0=\{y \in Y|\dim_y Y_{f(y)} = 0 \}$ is open in $Y$. Since $g_A^{-1}(U)$ contains all generic points of the closed fiber the closed fiber of $Y$ must have dimension $<1$. It follows that $U_0$ contains all of $Y \times_B \Spec {A/I}$ and therefore $Y \subset U_0$ because $\Spec A/I$ contains all closed points of $\Spec A$. This contradicts the fact that $C \to \Spec A$ is a relative curve. It follows that $g_A^{-1}(U)$ contains all the generic points of the generic fiber, as desired. It follows that the resulting $g_A: C \to X_A$ is a $A$-point of $CM_{X/B}$ which algebraizes $(h_i,C_i,g_i)$. 
\end{proof}

\begin{theorem}[Algebraicity of $\CMfunctor$]
\label{theorem-cm-general}
Let $S$ be a scheme. Let $f : X \to B$ be morphism of algebraic spaces
over $S$. Assume that $f$ is of finite presentation and separated. Then
$\CMfunctor_{X/B}$ is an algebraic space over $S$.
\end{theorem}

\begin{proof}
Set $\mathcal{X} = \CMfunctor_{X/B}$. We have seen that $\mathcal{X}$
is a sheaf over $(\Sch/S)_{fppf}$ with diagonal representable
by algebraic spaces
(Lemmas \ref{cm-rep-diagonal} and \ref{cm-is-sheaf}).
Hence it suffices to find a scheme $W$ and a surjective and \'etale
morphism $W \to \mathcal{X}$.

\medskip\noindent
Let $B'$ be a scheme and let $B' \to B$ be a surjective \'etale morphism.
Set $X' = B' \times_B X$ and denote $f' : X' \to B'$ the projection.
Then $\mathcal{X}' = \CMfunctor_{X'/B'}$ is equal to the fibre
product of $\mathcal{X}$ with $B'$ and $\mathcal{X}'\rightarrow\mathcal{X}$ is representable
(Remark \ref{remark-cm-base-change}). 
Moreover, because the morphism $\mathcal{X}' \to \mathcal{X}$ is the base change of a surjective and \'etale morphism, it too is surjective and \'etale.
Hence it suffices to prove the result for $\mathcal{X}'$.
In other words, we may assume $B$ is a scheme.

\medskip\noindent
Assume $B$ is a scheme. In this case we may replace $S$ by $B$, see
Algebraic Spaces, Section Tag 03I3.
Thus we may assume $S = B$.

\medskip\noindent
Assume $S = B$.
By Bootstrap, Lemma Tag 04SK, we may assume that $B$ is affine.

\medskip\noindent
Assume $S = B$ is affine, say $S = \Spec(\Lambda)$.
Write $\Lambda = \colim \Lambda_i$ as a filtered colimit with each $\Lambda_i$
of finite type over $\mathbf{Z}$. For some $i$ we can find
a morphism of algebraic spaces $X_i \to \Spec(\Lambda_i)$
which is separated and of finite presentation and whose base change
to $\Lambda$ is $X$. See Limits of Spaces, Lemmas
\ref{spaces-limits-lemma-descend-finite-presentation} and
\ref{spaces-limits-lemma-descend-separated-morphism}.
If we show that $\CMfunctor_{X_i/\Spec(\Lambda_i)}$ is an
algebraic space, then it follows by base change
(Remark \ref{remark-cm-base-change} and Section Tag 03I3)
that $\mathcal{X}$ is an algebraic space.
Thus we may assume that $\Lambda$ is a finite type $\mathbf{Z}$-algebra.

\medskip\noindent
Assume $S = B = \Spec(\Lambda)$ is affine of finite type over $\mathbf{Z}$.
In this case we will verify conditions (1), (2), and (3) of 
Proposition Tag 07Y1
to conclude that $\mathcal{X}$ is an algebraic space.

\medskip\noindent
Lemma \ref{cm-rep-diagonal} provides (1).

\medskip\noindent
To check (2) we have to verify axioms [-1], [0], [1], [2], [3], [4], and [5]
of Artin's Axioms, Section \ref{artin-section-axioms}.
We omit the verification of [-1] and axioms
[0], [1], [2], [3] correspond respectively to
Lemmas \ref{cm-is-sheaf},
\ref{lemma-cm-limits},
\ref{lemma-cm-RS-star},
\ref{lemma-cm-tangent-space}.
Condition (3) is Lemma \ref{lemma-cm-existence}.

\medskip\noindent
It remains to show condition [5] which is openness of versality.
To see this we will use
Artin's Axioms, Lemma \ref{artin-lemma-SGE-implies-openness-versality}.
We have already seen that $\mathcal{X}$ has diagonal
representable by algebraic spaces, has (RS*), and is limit preserving
(see lemmas used above).
Hence we only need to see that $\mathcal{X}$ satisfies the strong
formal effectiveness formulated in
Artin's Axioms, Lemma \ref{artin-lemma-SGE-implies-openness-versality}.
This is lemma \ref{lemma-cm-strong-effectiveness}.

\medskip\noindent
Note that $\Lambda$ is a G-ring, see
More on Algebra, Proposition \ref{more-algebra-proposition-ubiquity-G-ring}.
Hence all local rings of $S$ are G-rings. Thus (3) holds and the proof is complete.
\end{proof}
\begin{lemma}
	In Situation \ref{section-cm}, the diagonal
	\[\Delta:\CMfunctor_{X/B}\rightarrow\CMfunctor_{X/B}\times\CMfunctor_{X/B}\]
	is quasi-compact.
\end{lemma}
\begin{proof}
	
\end{proof}
\end{document}
