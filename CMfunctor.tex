\IfFileExists{stacks-project.cls}{%
\documentclass{stacks-project}
}{%
\documentclass{amsart}
}

% The following AMS packages are automatically loaded with
% the amsart documentclass:
%\usepackage{amsmath}
%\usepackage{amssymb}
%\usepackage{amsthm}

% For dealing with references we use the comment environment
\usepackage{verbatim}
\newenvironment{reference}{\comment}{\endcomment}
%\newenvironment{reference}{}{}
\newenvironment{slogan}{\comment}{\endcomment}
\newenvironment{history}{\comment}{\endcomment}

% For commutative diagrams you can use
% \usepackage{amscd}
\usepackage[all]{xy}

% We use 2cell for 2-commutative diagrams.
\xyoption{2cell}
\UseAllTwocells

% To put source file link in headers.
% Change "template.tex" to "this_filename.tex"
% \usepackage{fancyhdr}
% \pagestyle{fancy}
% \lhead{}
% \chead{}
% \rhead{Source file: \url{template.tex}}
% \lfoot{}
% \cfoot{\thepage}
% \rfoot{}
% \renewcommand{\headrulewidth}{0pt}
% \renewcommand{\footrulewidth}{0pt}
% \renewcommand{\headheight}{12pt}

\usepackage{multicol}

% For cross-file-references
\usepackage{xr-hyper}

% Package for hypertext links:
\usepackage{hyperref}

% For any local file, say "hello.tex" you want to link to please
% use \externaldocument[hello-]{hello}
\externaldocument[introduction-]{introduction}
\externaldocument[conventions-]{conventions}
\externaldocument[sets-]{sets}
\externaldocument[categories-]{categories}
\externaldocument[topology-]{topology}
\externaldocument[sheaves-]{sheaves}
\externaldocument[sites-]{sites}
\externaldocument[stacks-]{stacks}
\externaldocument[fields-]{fields}
\externaldocument[algebra-]{algebra}
\externaldocument[brauer-]{brauer}
\externaldocument[homology-]{homology}
\externaldocument[derived-]{derived}
\externaldocument[simplicial-]{simplicial}
\externaldocument[more-algebra-]{more-algebra}
\externaldocument[smoothing-]{smoothing}
\externaldocument[modules-]{modules}
\externaldocument[sites-modules-]{sites-modules}
\externaldocument[injectives-]{injectives}
\externaldocument[cohomology-]{cohomology}
\externaldocument[sites-cohomology-]{sites-cohomology}
\externaldocument[dga-]{dga}
\externaldocument[dpa-]{dpa}
\externaldocument[hypercovering-]{hypercovering}
\externaldocument[schemes-]{schemes}
\externaldocument[constructions-]{constructions}
\externaldocument[properties-]{properties}
\externaldocument[morphisms-]{morphisms}
\externaldocument[coherent-]{coherent}
\externaldocument[divisors-]{divisors}
\externaldocument[limits-]{limits}
\externaldocument[varieties-]{varieties}
\externaldocument[topologies-]{topologies}
\externaldocument[descent-]{descent}
\externaldocument[perfect-]{perfect}
\externaldocument[more-morphisms-]{more-morphisms}
\externaldocument[flat-]{flat}
\externaldocument[groupoids-]{groupoids}
\externaldocument[more-groupoids-]{more-groupoids}
\externaldocument[etale-]{etale}
\externaldocument[chow-]{chow}
\externaldocument[intersection-]{intersection}
\externaldocument[pic-]{pic}
\externaldocument[adequate-]{adequate}
\externaldocument[dualizing-]{dualizing}
\externaldocument[duality-]{duality}
\externaldocument[discriminant-]{discriminant}
\externaldocument[local-cohomology-]{local-cohomology}
\externaldocument[curves-]{curves}
\externaldocument[resolve-]{resolve}
\externaldocument[models-]{models}
\externaldocument[pione-]{pione}
\externaldocument[etale-cohomology-]{etale-cohomology}
\externaldocument[proetale-]{proetale}
\externaldocument[crystalline-]{crystalline}
\externaldocument[spaces-]{spaces}
\externaldocument[spaces-properties-]{spaces-properties}
\externaldocument[spaces-morphisms-]{spaces-morphisms}
\externaldocument[decent-spaces-]{decent-spaces}
\externaldocument[spaces-cohomology-]{spaces-cohomology}
\externaldocument[spaces-limits-]{spaces-limits}
\externaldocument[spaces-divisors-]{spaces-divisors}
\externaldocument[spaces-over-fields-]{spaces-over-fields}
\externaldocument[spaces-topologies-]{spaces-topologies}
\externaldocument[spaces-descent-]{spaces-descent}
\externaldocument[spaces-perfect-]{spaces-perfect}
\externaldocument[spaces-more-morphisms-]{spaces-more-morphisms}
\externaldocument[spaces-flat-]{spaces-flat}
\externaldocument[spaces-groupoids-]{spaces-groupoids}
\externaldocument[spaces-more-groupoids-]{spaces-more-groupoids}
\externaldocument[bootstrap-]{bootstrap}
\externaldocument[spaces-pushouts-]{spaces-pushouts}
\externaldocument[groupoids-quotients-]{groupoids-quotients}
\externaldocument[spaces-more-cohomology-]{spaces-more-cohomology}
\externaldocument[spaces-simplicial-]{spaces-simplicial}
\externaldocument[spaces-duality-]{spaces-duality}
\externaldocument[formal-spaces-]{formal-spaces}
\externaldocument[restricted-]{restricted}
\externaldocument[spaces-resolve-]{spaces-resolve}
\externaldocument[formal-defos-]{formal-defos}
\externaldocument[defos-]{defos}
\externaldocument[cotangent-]{cotangent}
\externaldocument[examples-defos-]{examples-defos}
\externaldocument[algebraic-]{algebraic}
\externaldocument[examples-stacks-]{examples-stacks}
\externaldocument[stacks-sheaves-]{stacks-sheaves}
\externaldocument[criteria-]{criteria}
\externaldocument[artin-]{artin}
\externaldocument[quot-]{quot}
\externaldocument[stacks-properties-]{stacks-properties}
\externaldocument[stacks-morphisms-]{stacks-morphisms}
\externaldocument[stacks-limits-]{stacks-limits}
\externaldocument[stacks-cohomology-]{stacks-cohomology}
\externaldocument[stacks-perfect-]{stacks-perfect}
\externaldocument[stacks-introduction-]{stacks-introduction}
\externaldocument[stacks-more-morphisms-]{stacks-more-morphisms}
\externaldocument[stacks-geometry-]{stacks-geometry}
\externaldocument[moduli-]{moduli}
\externaldocument[moduli-curves-]{moduli-curves}
\externaldocument[examples-]{examples}
\externaldocument[exercises-]{exercises}
\externaldocument[guide-]{guide}
\externaldocument[desirables-]{desirables}
\externaldocument[coding-]{coding}
\externaldocument[obsolete-]{obsolete}
\externaldocument[fdl-]{fdl}
\externaldocument[index-]{index}

% Theorem environments.
%
\theoremstyle{plain}
\newtheorem{theorem}[subsection]{Theorem}
\newtheorem{proposition}[subsection]{Proposition}
\newtheorem{lemma}[subsection]{Lemma}

\theoremstyle{definition}
\newtheorem{definition}[subsection]{Definition}
\newtheorem{example}[subsection]{Example}
\newtheorem{exercise}[subsection]{Exercise}
\newtheorem{situation}[subsection]{Situation}

\theoremstyle{remark}
\newtheorem{remark}[subsection]{Remark}
\newtheorem{remarks}[subsection]{Remarks}

\numberwithin{equation}{subsection}

% Macros
%
\def\lim{\mathop{\rm lim}\nolimits}
\def\colim{\mathop{\rm colim}\nolimits}
\def\Spec{\mathop{\rm Spec}}
\def\Hom{\mathop{\rm Hom}\nolimits}
\def\Ext{\mathop{\rm Ext}\nolimits}
\def\SheafHom{\mathop{\mathcal{H}\!{\it om}}\nolimits}
\def\SheafExt{\mathop{\mathcal{E}\!{\it xt}}\nolimits}
\def\Sch{\textit{Sch}}
\def\Mor{\mathop{\rm Mor}\nolimits}
\def\Ob{\mathop{\rm Ob}\nolimits}
\def\Sh{\mathop{\textit{Sh}}\nolimits}
\def\NL{\mathop{N\!L}\nolimits}
\def\proetale{{pro\text{-}\acute{e}tale}}
\def\etale{{\acute{e}tale}}
\def\QCoh{\textit{QCoh}}
\def\Ker{\mathop{\rm Ker}}
\def\Im{\mathop{\rm Im}}
\def\Coker{\mathop{\rm Coker}}
\def\Coim{\mathop{\rm Coim}}

%
% Macros for moduli stacks/spaces
%
\def\QCohstack{\mathcal{QC}\!{\it oh}}
\def\Cohstack{\mathcal{C}\!{\it oh}}
\def\Spacesstack{\mathcal{S}\!{\it paces}}
\def\Quotfunctor{{\rm Quot}}
\def\Hilbfunctor{{\rm Hilb}}
\def\Curvesstack{\mathcal{C}\!{\it urves}}
\def\Polarizedstack{\mathcal{P}\!{\it olarized}}
\def\Complexesstack{\mathcal{C}\!{\it omplexes}}
% \Pic is the operator that assigns to X its picard group, usage \Pic(X)
% \Picardstack_{X/B} denotes the Picard stack of X over B
% \Picardfunctor_{X/B} denotes the Picard functor of X over B
\def\Pic{\mathop{\rm Pic}\nolimits}
\def\Picardstack{\mathcal{P}\!{\it ic}}
\def\Picardfunctor{{\rm Pic}}
\def\Deformationcategory{\mathcal{D}\!{\it ef}}
\def\CMfunctor{\text{CM}}



% OK, start here.
%
\begin{document}

\title{The CM functor}


\maketitle


\section{The (Honsen?) functor}
\label{section-cm}

\noindent
In this section we prove the (Honsen?) functor is an algebraic space.\\

\begin{situation}
\label{situation-cm}
Let $S$ be a scheme. Let $f : X \to B$ be a morphism of algebraic spaces
over $S$. Assume that $f$ is of separated and of finite presentation.
For any scheme $T$ over $B$ we will denote $X_T$ the base change
of $X$ to $T$. Given such a $T$ we set:
\begin{align*}
\CMfunctor_{X/B}(T) = & (h, C \to T, g)  \text{ where } 
h \text{ is a morphism } h : T \to B \text{ and } g \text{ is a morphism } g : C \to X \\ & \text{satisfying the following properties}
\end{align*}
\begin{enumerate}
\item The diagram below is commutative: 
$$
\xymatrix{
C \ar[r]_g \ar[d] & X \ar[d] \\
T \ar[r]^h & B
}
$$
\item $C$ is an algebraic space and $C \to T$ is a Cohen-Macauley morphism of relative dimension 1 which is flat, proper, and of finite presentation.
\item The induced morphism $(g?) : C \to X_T$ is finite
\item There is an open $U \subset X_T$ where
\begin{enumerate}
\item $(g?)^{-1}(U) \to U$ is a closed immersion
\item For every $t \in T$, the fiber
$\left((g?)^{-1}(U)\right)_t$ is dense in $C_t$
\end{enumerate}
\end{enumerate}
\end{situation}

Effectiveness of formal objects:

\begin{theorem} Fix a complete Noetherian local $\mathcal{O}_S$-algebra $A$ and a collection of $\Spec A/m_i$ points of $CM_{X/B}$ for all $i$ which are compatible, call it $(h_i,C_i,g_i)$. Then there exists a $\Spec A$ point of $CM_{X/B}$ which restricts to $(h_i,C_i,g_i)$. 
\end{theorem}

\begin{proof}

We view the data of $(h,C, g)$ as a coherent algebra on $X_T$ by considering $(g_T)_*\mathcal{O}_{X_T}$. Conversely, given a coherent algebra $\mathcal{A}$ on $X_T$ that is $T$-flat with proper support over $T$, we take the relative spectrum of $\mathcal{A}$. Let $(h_i,C_i,g_i)$ be a compatible system of $\Spec A/m^n$ points of $CM_{X/B}$ and consider the corresponding coherent algebras $\mathcal{A}_i$ on $X_i=X \times_B  \Spec A/m^i$. 
The underlying coherent sheaf and its algebra structure can be described by morphisms $\hat{\mathcal{A}} \otimes \hat{\mathcal{A}} \to \hat{\mathcal{A}}$ and the unit $\mathcal{O}_{\hat{X}} \to \hat{\mathcal{A}}$. That this yields an algebra structure is expressed by the commutativity of certain diagrams. By Lemma 08BE these diagrams correspond to diagrams in $Coh_{X_A/\Spec A}$. As such we obtain a coherent algebra $\mathcal{A}$ on $X_A=X \times_B \Spec A$ whose relative spectrum is a candidate for algebraizing the formal point $(h_i,C_i,g_i)$ corresponding to $(\mathcal{A}_i)$. 
\\

It remains to show that $C=\underline{\Spec}_X(\mathcal{A})$ is a $\Spec A$-point of $CM_{X/B}$. Since it arises as the relative spectrum of a coherent algebra which is $\Spec A$-flat with proper support over $\Spec A$, $C \to X_{\Spec A}$ is finite with proper support of finite presentation. Since $C_i$ is $\Spec A/m^i$-flat for each $i$ Lemma 0523 shows $C$ is $\Spec A$-flat. To verify that $g_A: C \to X_A$ is a Cohen-Maucalay morphism use the fact that the closed fiber of $C \times_B \Spec A/m \to \Spec A/m$ is a Cohen-Maucalay morphism and Lemma 045U to conclude. 
\\

It remains to show the locus where the morphism $C_T \to X_A$ is a closed immersion is dense on the fibers. Note that this locus is open as it can be identified with the complement of $\text{Supp}(\text{Coker}[O_{X_A} \to \mathcal{A}])$ and the latter is closed because it is the support of a coherent sheaf. Since pullback is right exact and taking supports with restriction to a closed subscheme (Lemma 00L3) we obtain 

\[\text{Supp}(\text{Coker}[O_{X_A} \to \mathcal{A}])|_{\Spec A/m}=\text{Supp}(\text{Coker}[O_{X_{A/m}} \to \mathcal{A}_0]))\]

and the latter is dense when pulled back to $C_0=C \times \Spec A/m$. This implies $g_A^{-1}(U)$ is dense in the closed fiber $C_0 \subset C$ and now we must conclude that it is dense in every fiber of $C$. Let $p \subset A$ be a prime ideal, by modding out and pulling back $g_A: C \to X_A$ we may suppose that $p=(0)$ is the generic point of $A$. 
\\

Suppose there is a generic point of the generic fiber $\eta \in C \times_B \Spec A_{(0)}$ not in $g_A^{-1}(U)$ then none of its specializations lies in $g_A^{-1}(U)$. Consider the composition $Y=\overline{\{\eta\}} \subset X_A \to \Spec A$ where $Y$ has the reduced induced structure. Then by Lemma 02FZ the set $U_0=\{y \in Y|\dim_y Y_{f(y)} = 0 \}$ is open in $Y$. Since $g_A^{-1}(U)$ contains all generic points of the closed fiber we see that $U_0$ contains all of $Y \times_B \Spec {A/m}$ and therefore $U_0=Y$, a contradiction since $C \to \Spec A$ is a relative curve. It follows that $g_A^{-1}(U)$ contains all the generic points of the generic fiber, as desired. It follows that the resulting $g_A: C \to X_A$ is a $A$-point of $CM_{X/B}$ which algebraizes $(h_i,C_i,g_i)$. 

\end{proof}


\end{document}